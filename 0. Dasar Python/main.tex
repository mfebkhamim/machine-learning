\documentclass[a4paper,12pt]{article}

\usepackage[margin=2.5cm, top=2cm]{geometry}   % atur margin + jarak atas
\usepackage{setspace}
\usepackage{xcolor}
\usepackage{listings}
\usepackage{titlesec}
\usepackage{array}
\usepackage{tabularx}
\usepackage{booktabs}
\usepackage[utf8]{inputenc}
\usepackage[T1]{fontenc}

% Pengaturan warna untuk kode Python
\definecolor{codegreen}{rgb}{0,0.6,0}
\definecolor{codegray}{rgb}{0.5,0.5,0.5}
\definecolor{codepurple}{rgb}{0.58,0,0.82}
\definecolor{backcolour}{rgb}{0.95,0.95,0.92}

\renewcommand{\tabularxcolumn}[1]{m{#1}}

\newcolumntype{Y}{>{\centering\arraybackslash}X}

\lstset{
    language=Python,
    backgroundcolor=\color{backcolour},
    commentstyle=\color{codegreen},
    keywordstyle=\color{magenta},
    numberstyle=\tiny\color{codegray},
    stringstyle=\color{codepurple},
    basicstyle=\ttfamily\small,
    breaklines=true,
    showstringspaces=false
}

\titleformat{\title}{\normalfont\Large\bfseries}{}{0pt}{}
\titlespacing*{\section}{0pt}{0.8\baselineskip}{0.6\baselineskip}

\title{\textbf{Belajar Python}}
\author{Mohammad Febryan Khamim}
\date{} 


% Isi Dokumen

\begin{document}

\maketitle
\vspace{-0.5\baselineskip}

\section{Perbedaan List, Tuple, Set dan Dictionary}

\subsection*{1. List}
\textbf{List} adalah tipe data yang ditulis berurutan yang berisi lebih dari satu tipe data.
\begin{itemize}
    \item \textbf{Contoh:} \lstinline|a = ["teks", 7, 2.5, True]|
    \item \textbf{Bersifat \textit{mutable}:} Bisa ditambahkan atau dikurangi.
    \begin{itemize}
        \item \lstinline|.append()| $\rightarrow$ ditambahkan. Contoh: \lstinline|a.append("satu")|
        \item \lstinline|del| $\rightarrow$ dihapus. Contoh: \lstinline|del a[2]|
    \end{itemize}
\end{itemize}

\subsection*{2. Tuple}
\textbf{Tuple} mirip seperti list tapi isinya \textbf{\textit{immutable}}, tidak dapat diubah nilainya.

\subsection*{3. Set}
\textbf{Set} adalah kumpulan item yang bersifat unik dan tak berurutan.
\begin{itemize}
    \item \textbf{Contoh:} \lstinline|S = {1, 2, "tuple"}|
    \item Jika terdapat lebih dari satu data yang sama $\rightarrow$ hanya tersimpan 1.
    \begin{lstlisting}
S = {1, 2, 2, 4}
print(S)  # Output: {1, 2, 4}
    \end{lstlisting}
    \item \textbf{Cara menambahkan data:} Harus menggunakan fungsi \lstinline|.add()|.
    \begin{lstlisting}
S = {1, 2, 2, 4}
S.add(0)
print(S)  # Output: {0, 1, 2, 4}
    \end{lstlisting}
\end{itemize}

\subsection*{4. Dictionary}
\textbf{Dictionary} adalah kumpulan pasangan kunci dan nilai (\textit{key : value}) dan tak berurutan.
\begin{itemize}
    \item \textbf{Format:} \lstinline|d = {"key" : "Value"}|
    \item \textbf{Contoh:}
    \begin{lstlisting}
d = {"hewan" : "gajah", "angka" : 1}
    \end{lstlisting}
    \item \textbf{Cara mengubah nilai:} 
    \begin{lstlisting}
d["hewan"] = "unta"
    \end{lstlisting}
\end{itemize}

\newpage

\textbf{Tabel Perbedaan}

Berikut ini adalah tabel perbedaan dari List, Tuple, Set, dan Dictionary.

\begin{tabularx}{\textwidth}{@{} 
    >{\centering\arraybackslash}m{2.5cm}
    >{\centering\arraybackslash}m{2.5cm}
    >{\centering\arraybackslash}m{2cm}
    >{\centering\arraybackslash}m{2cm}
    Y
@{}}
\toprule
\textbf{Fitur} & \textbf{List} & \textbf{Tuple} & \textbf{Set} & \textbf{Dictionary} \\ \midrule

\textbf{Simbol} 
& \texttt{[ ]} 
& \texttt{( )} 
& \texttt{\{ \}} 
& \texttt{\{k:v\}} \\ \addlinespace

\textbf{Sifat} 
& \textit{Mutable} 
& \textit{Immutable} 
& \textit{Mutable} 
& \textit{Mutable} \\ \addlinespace

\textbf{Urutan} 
& \textit{Ordered} 
& \textit{Ordered} 
& \textit{Unordered} 
& \textit{Unordered} \\ \addlinespace

\textbf{Duplikasi} 
& Boleh ada data sama 
& Boleh ada data sama 
& \textbf{Tidak Boleh} (Unik) 
& Key harus unik, Value boleh sama \\ \addlinespace

\textbf{Contoh} 
& \texttt{[1, 2, 2]} 
& \texttt{(1, 2, 2)} 
& \texttt{\{1, 2\}} 
& \texttt{\{"a": 1\}} \\

\bottomrule
\end{tabularx}

\vspace{0.5cm}

% Bagian 2 : 

\section{IF-ELSE Statement}
Dalam Python, terdapat cara untuk menjelaskan terkait \textit{conditional statement}, yakni menggunakan IF ELSE. 
Berikut ini adalah penjelasannya. 

\begin{enumerate}
    \item \textbf{Satu Kondisi}
    
    Bagian ini digunakan untuk dapat menjalankan suatu perintah ketika sebuah syarat terpenuhi, berikut ini contohnya. 
    \begin{lstlisting}[language=Python]
    nilai = 80
    if nilai >= 75:
        print("Selamat, Anda Lulus!")
    \end{lstlisting}

    \item \textbf{Banyak Kondisi}

    Apabila ingin menjalankan suatu perintah yang memiliki beberapa persyaratan, berikut ini contohnya. 
    \begin{lstlisting}[language=Python]
    skor = 85
    
    if skor >= 90:
        print("Predikat: A")
    elif skor >= 80:
        print("Predikat: B")
    elif skor >= 70:
        print("Predikat: C")
    else:
        print("Predikat: D")
    \end{lstlisting}

    \item \textbf{Operator Perbandingan dan Logika}

    Berikut ini adalah beberapa operator logika yang sering digunakan dalam IF ELSE dalam Python. 
    Kondisi di dalam \texttt{if} sering kali menggunakan operator berikut:
    \begin{itemize}
        \item \textbf{Perbandingan:} $==, \neq, >, <, \geq, \leq$
        \item \textbf{Logika:} 
            \begin{itemize}
                \item \texttt{and} : Benar jika kedua kondisi terpenuhi ($A \land B$).
                \item \texttt{or} : Benar jika salah satu kondisi terpenuhi ($A \lor B$).
            \end{itemize}
    \end{itemize}
\end{enumerate}

\newpage

\section{Looping dalam Python}

Untuk membuat perulangan atau \textit{looping} pada Python, terdapat beberapa hal utama yang dapat digunakan dalam Python, di antaranya adalah : 

\begin{enumerate}
    \item \textbf{For Loop}

    \textbf{For Loop} digunakan untuk melakukan iterasi di antara beberapa barisan, seperti \textit{list}, \textit{tuple}, \textit{string}, atau \textit{range}. 
    Jumlah iterasinya \textbf{sudah diketahui} dari awal. 

    \begin{lstlisting}[language=Python, caption=Contoh For Loop]
    # Iterasi menggunakan range
    for i in range(1, 6):
        print(f"Perulangan ke-{i}")
    
    # Iterasi pada list
    buah = ["apel", "mangga", "pisang"]
    for item in buah:
        print(f"Saya suka {item}")
    \end{lstlisting}

    \item \textbf{While Loop}

    Untuk \textbf{While Loop} digunakan untuk melakukan perulangan selama masih memenuhi kondisi tertentu (atau kondisi tersebut bernilai \textbf{True}).

    \begin{lstlisting}[language=Python, caption=Contoh While Loop]
count = 0
while count < 5:
    print(f"Nilai count: {count}")
    count += 1  # Increment 
    \end{lstlisting}

    \item \textbf{Break dan Continue}

    Pernyataan \textbf{kontrol perulangan} (\textit{loop control statements}) mengubah eksekusi dari urutan normalnya. 

    \begin{itemize}
        \item \textbf{Break}. 
        Pernyataan \texttt{break} digunakan untuk menghentikan perulangan secara paksa, bahkan jika kondisi \textit{loop} sebenarnya masih bernilai \textit{True}. Setelah \texttt{break} dipanggil, program akan langsung menjalankan baris kode setelah blok perulangan.
        \begin{lstlisting}[language=Python, caption=Contoh Break]
    for i in range(1, 10):
        if i == 5:
            break  # Loop berhenti total saat i bernilai 5
        print(i)
    # Output: 1, 2, 3, 4
        \end{lstlisting}

        \item \textbf{Continue}. 
        Pernyataan \texttt{continue} digunakan untuk melompati sisa kode di dalam iterasi saat ini dan langsung berpindah ke iterasi berikutnya. Perulangan tidak berhenti, hanya "loncat" ke pengecekan kondisi berikutnya.

\begin{lstlisting}[language=Python, caption=Contoh Continue]
    for i in range(1, 6):
        if i == 3:
            continue  # Angka 3 akan dilewati
        print(i)
    # Output: 1, 2, 4, 5
\end{lstlisting}

    \item \textbf{Pass}.
    Pernyataan \texttt{pass} adalah pernyataan \textit{null} (kosong).
    Perbedaannya dengan komentar adalah Python tidak mengabaikan \texttt{pass}, tetapi secara sintaksis \texttt{pass} digunakan sebagai penampung (\textit{placeholder}) saat kode belum ditulis, tetapi blok tersebut wajib ada secara struktur.

    \begin{lstlisting}[language=Python, caption=Contoh Pass]
for i in range(5):
    if i == 2:
        pass  # Tidak melakukan apa-apa
    print(f"Angka: {i}")
    \end{lstlisting}
    
    \end{itemize}

    \item \textbf{Nested Loop}
    Perulangan di dalam perulangan lainnya.

    \begin{lstlisting}[language=Python, caption=Contoh Nested Loop]
for i in range(1, 4):
    for j in range(1, 4):
        print(f"i={i}, j={j}")
    \end{lstlisting}
    
\end{enumerate}

\newpage
\setcounter{section}{0}

\begin{center}
    % Mengatur ukuran teks menjadi sangat besar dan tebal
    {\Huge \textbf{Library dalam Python}} \\[0.25cm]
    % Menambahkan sub-judul atau nama jika diperlukan
    8 Januari 2026
\end{center}

\section{Library Pandas}

\textbf{Library Pandas} merupakan \textit{open source} yang menyediakan beberapa peralatan untuk kebutuhan analisis, manipulasi, dan pembersihan data. 
Pandas memiliki format data yang disebut DataFrame untuk membentuknya dalam struktur data 2 dimensi atau tabel. 
Berikut ini adalah beberapa contoh \textit{syntax} yang umum digunakan dalam Pandas. 

\begin{itemize}
    \item \textbf{Syntax mengubah \textit{dictionary} jadi tabel}

    \begin{lstlisting}[language=Python, caption=Dictionary jadi Tabel]
# Membuat data dictionary
data = {
    'Nama': ['Anna', 'Bob', 'Charlie', 'David', 'Eva'],
    'Umur': [18, 19, 17, 18, 19],
    'Matematika': [85, 90, 78, 92, 88],
    'Bahasa_Inggris': [88, 85, 92, 89, 94],
    'IPA': [90, 87, 84, 88, 91]
}

# Membuat DataFrame dari data
df = pd.DataFrame(data)

# Menampilkan DataFrame
print("DataFrame Awal:")
print(df)

    \end{lstlisting}

    \item \textbf{Membaca dan Menyimpan Data}

    \begin{lstlisting}[language=Python]
# Membaca Data
df = pd.read_csv('data.csv')
df_excel = pd.read_excel('data.xlsx')

# Menyimpan dictionary dalam CSV
df.to_csv('output.csv', index=False)
df.to_excel('output.xlsx', index=False)
\end{lstlisting}

Adanya \textit{syntax} berupa \textbf{index} tersebut menunjukkan apakah indeks (urutan baris) dimasukkan ke dalam \textit{file} yang akan disimpan atau tidak. 

\item \textbf{Melihat Struktur Data}

\begin{lstlisting}[language=Python]
df.head()        # 5 baris pertama
df.tail()        # 5 baris terakhir
df.info()        # informasi kolom & tipe data
df.describe()    # statistik deskriptif
df.shape         # jumlah baris dan kolom
\end{lstlisting}

\newpage

\item \textbf{Seleksi Baris dan Kolom}
\begin{lstlisting}[language=Python]
df['Nama']
df[['Nama', 'Umur']]

df.loc[0]          # berdasarkan index label
df.iloc[0]         # berdasarkan posisi

df.loc[0, 'Nama']  # sel tertentu
\end{lstlisting}

\item \textbf{Filtering Data}
\begin{lstlisting}[language=Python]
df[df['Umur'] > 20]
df[df['Kota'] == 'Bandung']
df[(df['Umur'] > 20) & (df['Kota'] == 'Bandung')]
\end{lstlisting}

\item \textbf{Mengatasi \textit{Missing Value}}

\begin{lstlisting}[language=Python]
df.isnull().sum()
df_drop = df.dropna()
df_fill = df.fillna(0)
\end{lstlisting}

\item \textbf{Manipulasi Kolom}
\begin{lstlisting}[language=Python]
df['Total'] = df['Nilai1'] + df['Nilai2']
df.rename(columns={'Nilai1':'Nilai_A'}, inplace=True)
df.drop(columns=['Nilai2'], inplace=True)
\end{lstlisting}
    
\end{itemize}

\section{Library PyTorch}

PyTorch adalah \textit{framework} Machine Learning (ML) \textit{open-source} yang dikembangkan untuk \textit{deep learning}.
\textit{Framework} ini sangat populer karena mendukung komputasi berbasis GPU dan dibuat dengan konsep \textit{dynamic computation graph}, di mana grafik komputasi dibentuk secara dinamis saat program dijalankan.

\subsection*{Fitur Utama PyTorch}
\begin{itemize}
    \item \textbf{Dynamic Graphs}: Mempermudah eksekusi dan proses \textit{debug}.
    \item \textbf{Automatic Diff. Engine}: Mempermudah komputasi gradien secara otomatis.
    \item \textbf{Mendukung CUDA}: Memungkinkan komputasi pada GPU untuk mempercepat proses pelatihan.
\end{itemize}

\subsection*{Komponen Utama}
Berikut ini terdapat beberapa komponen yang dapat digunakan pada PyTorch, di antaranya sebagai berikut. 

\begin{enumerate}
    \item \textbf{Tensor}

    Tensor merupakan struktur data inti di PyTorch yang serupa dengan \textit{array} pada NumPy, tetapi memiliki kemampuan akselerasi pada GPU.
    Dalam \textit{deep learning}, Tensor digunakan untuk menyimpan data \textit{input}, \textit{output}, dan parameter model.
    Tensor digunakan untuk mendukung diferensiasi otomatis yang cocok untuk tugas \textit{deep learning}. 

    \begin{lstlisting}[language=Python]
import torch
t = torch.tensor([[1, 2, 3, 4],
                 [5, 6, 7, 8],
                 [9, 10, 11, 12]])

print("Reshaping")
print(t.reshape(6, 2))

print("\nResizing")
print(t.view(2, 6))

print("\nTransposing")
print(t.transpose(0, 1))
    \end{lstlisting}

    \item \textbf{Autograd dan Computational Graphs}

    Autograd secara otomatis dapat digunakan untuk menghitung turunan atau gradien dari suatu operasi yang sangat berguna dalam \textit{backpropagation} pada Neural Network. 
    Sementara itu, Computational Graph merupakan representasi grafis dari operasi matematis yang dilakukan pada tensor. 

    \begin{lstlisting}[language=Python]
import torch
import torch.nn as nn


class NeuralNetwork(nn.Module):
    def __init__(self):
        super(NeuralNetwork, self).__init__()
        self.fc1 = nn.Linear(10, 16)
        self.fc2 = nn.Linear(16, 8)
        self.fc3 = nn.Linear(8, 1)

    def forward(self, x):
        x = torch.relu(self.fc1(x))
        x = torch.relu(self.fc2(x))
        x = torch.sigmoid(self.fc3(x))
        return x


model = NeuralNetwork()
print(model)
    \end{lstlisting}
    
    \item \textbf{Loss Function dan Optimizer}
    \begin{itemize}
        \item \textbf{Loss Function}: Digunakan untuk menghitung tingkat kesalahan (\textit{error}) antara prediksi model dengan label asli.
        \item \textbf{Optimizer}: Bertugas memperbarui bobot model berdasarkan perhitungan gradien untuk meminimalkan \textit{loss}.
    \end{itemize}

    \item \textbf{Proses Pelatihan Model}

    Terdapat empat tahapan utama dalam pelatihan model di PyTorch.
    \begin{enumerate}
        \item \textbf{Forward Pass}: Mengalirkan data melalui model untuk mendapatkan prediksi.
        \item \textbf{Perhitungan Loss}: Mengukur seberapa jauh hasil prediksi dari target.
        \item \textbf{Backward Pass}: Menghitung gradien \textit{loss} terhadap parameter model.
        \item \textbf{Pembaruan Parameter}: Menggunakan \textit{optimizer} untuk memperbarui bobot.
    \end{enumerate}

    \item \textbf{Model Deep Learning pada \textit{pytorch}}

    Beberapa model populer yang didukung antara lain.
    \begin{itemize}
        \item \textbf{Convolutional Neural Network (CNN)}: Fokus pada pemrosesan data gambar.
        \item \textbf{Recurrent Neural Network (RNN)}: Fokus pada data urutan (\textit{sequence}) seperti teks.
        \item \textbf{Model Generative}: Seperti GAN (\textit{Generative Adversarial Networks}) dan VAE (\textit{Variational Autoencoders}).
    \end{itemize}

    \item \textbf{Contoh Implementasi Kode}

    Berikut adalah contoh pembuatan Tensor dan operasi dasar di PyTorch.

    \begin{lstlisting}[language=Python, caption=Operasi Dasar Tensor PyTorch]
import torch

# Membuat tensor dari list
data = [[1, 2], [3, 4]]
x_data = torch.tensor(data)

# Membuat tensor acak
shape = (2, 3,)
rand_tensor = torch.rand(shape)

# Operasi matematika sederhana
x = torch.tensor([5.0], requires_grad=True)
y = x**2
y.backward() # Menghitung gradien

print(f"Tensor Data: \n {x_data}")
print(f"Gradien dari x^2 pada x=5: {x.grad}") # Hasilnya 2*x = 10
    \end{lstlisting}

Berikut adalah contoh kode Python untuk mendefinisikan model, menghitung \textit{loss}, dan melakukan pembaruan bobot (optimasi).

\begin{lstlisting}[language=Python, caption=Neural Network Sederhana dengan PyTorch]
import torch
import torch.nn as nn
import torch.optim as optim

# 1. Menyiapkan data dummy (Input & Target)
# Misal: 10 data dengan 5 fitur masing-masing
input_data = torch.randn(10, 5)
target = torch.randn(10, 1)

# 2. Mendefinisikan Arsitektur Model
class SimpleNet(nn.Module):
    def __init__(self):
        super(SimpleNet, self).__init__()
        # Layer input ke hidden (5 fitur ke 10 neuron)
        self.hidden = nn.Linear(5, 10)
        # Fungsi aktivasi
        self.relu = nn.ReLU()
        # Layer hidden ke output (10 neuron ke 1 output)
        self.output = nn.Linear(10, 1)

    def forward(self, x):
        x = self.hidden(x)
        x = self.relu(x)
        x = self.output(x)
        return x

# 3. Inisialisasi Model, Loss Function, dan Optimizer
model = SimpleNet()
criterion = nn.MSELoss() # Mean Squared Error
optimizer = optim.SGD(model.parameters(), lr=0.01)

# 4. Proses Pelatihan Singkat (5 Epoch)
for epoch in range(5):
    # Forward pass
    prediction = model(input_data)
    loss = criterion(prediction, target)
    
    # Backward pass & Optimization
    optimizer.zero_grad() # Reset gradien
    loss.backward()       # Backpropagation
    optimizer.step()      # Update bobot
    
    print(f"Epoch {epoch+1}, Loss: {loss.item():.4f}")
\end{lstlisting}

\textbf{Penjelasan Komponen}
\begin{itemize}
    \item \textbf{\texttt{nn.Module}}: Kelas dasar untuk semua modul jaringan saraf di PyTorch.
    \item \textbf{\texttt{forward()}}: Menentukan bagaimana data mengalir melalui lapisan-lapisan \textit{neural network}.
    \item \textbf{\texttt{optimizer.zero\_grad()}}: Sangat penting untuk membersihkan gradien lama sebelum menghitung gradien baru pada setiap iterasi agar tidak terakumulasi.
    \item \textbf{\texttt{loss.backward()}}: Langkah di mana PyTorch menghitung gradien untuk setiap parameter yang memiliki \texttt{requires\_grad=True}.
\end{itemize}
    
\end{enumerate}

\section{Library TensorFlow}

TensorFlow adalah sebuah \textbf{framework open-source} untuk \textit{machine learning} dan \textit{artificial intelligence} yang dikembangkan oleh Google Brain. 
\textit{Framework} ini dibuat untuk membangun dan melatih model ML dan \textit{deep learning}. 

Dalam penggunaan sehari-hari dengan Python, TensorFlow menyediakan ekosistem yang lengkap termasuk:
\begin{itemize}
    \item API tingkat rendah untuk mendefinisikan dan menjalankan komputasi numerik.
    \item Integrasi dengan \textbf{Keras}, yaitu API tingkat tinggi untuk membangun model neural network secara intuitif.
    \item Dukungan alat visualisasi dan debugging seperti TensorBoard.
    \item Mekanisme optimasi model dan \textit{deployment} untuk perangkat \textit{mobile} dan \textit{web} (mis.\ TensorFlow Lite dan TensorFlow.js). 
\end{itemize}

TensorFlow memungkinkan penggunaan pada berbagai perangkat komputasi seperti CPU, GPU, dan TPU untuk mempercepat proses pelatihan model.

\vspace{0.3cm}

\textbf{Kelebihan TensorFlow} di antaranya sebagai berikut. 
\begin{itemize}
    \item \textbf{Skalabilitas tinggi} untuk riset dan produksi.
    \item \textbf{Ekosistem lengkap} termasuk deployment, model pretrained, dan alat bantu lain
    \item Mendukung komputasi terdistribusi dan akselerator perangkat keras.
    \item Ketersediaan \textbf{API dalam Python} yang kuat dan fleksibel.
\end{itemize}

\textbf{Perbandingan Framework ML/DL}

Berikut ini adalah tabel perbandingan antara empat \textit{framework} / \textit{library} populer dalam \textit{domain} \textit{machine learning} dan \textit{deep learning}:

\begin{tabularx}{\textwidth}{@{} 
    >{\centering\arraybackslash}m{3cm}
    >{\centering\arraybackslash}m{3cm}
    >{\centering\arraybackslash}m{3cm}
    >{\centering\arraybackslash}m{3cm}
    Y
@{}}
\toprule
\textbf{Fitur} & \textbf{TensorFlow} & \textbf{PyTorch} & \textbf{Keras} & \textbf{Scikit-Learn} \\ \midrule

\textbf{Jenis Library}
& Deep Learning
& Deep Learning
& High-level DL API
& Machine Learning klasik \\ \addlinespace

\textbf{Tingkat Abstraksi}
& Rendah–tinggi
& Rendah–menengah
& Tinggi
& Tinggi \\ \addlinespace

\textbf{Kemudahan Penggunaan}
& Cukup kompleks
& Lebih natural (Pythonic)
& Sangat mudah
& Sangat mudah \\ \addlinespace

\textbf{Graf Komputasi}
& Static + Eager Execution
& Dynamic (define-by-run)
& Menggunakan backend TensorFlow
& Tidak menggunakan computational graph \\ \addlinespace

\textbf{Fokus Utama}
& Produksi dan deployment skala besar
& Riset dan prototyping
& Pembangunan model cepat
& Algoritma ML tradisional \\ \addlinespace

\textbf{Dukungan Deployment}
& Sangat kuat (TF Lite, TF Serving, TF.js)
& Terbatas dibanding TF
& Mengikuti TensorFlow
& Tidak berfokus pada deployment \\ \addlinespace

\textbf{Contoh Use Case}
& CNN, RNN, NLP, Vision
& Riset DL, GAN, NLP
& Prototyping neural network
& Klasifikasi, regresi, clustering \\

\bottomrule
\end{tabularx}

\newpage

\textbf{Catatan:}
\begin{itemize}
    \item Keras pada awalnya adalah API tinggi yang berjalan di atas TensorFlow, sehingga memudahkan pembangunan model DL tanpa perlu detail komputasi tingkat rendah.
    \item Scikit-Learn fokus pada algoritma\textit{ machine learning} klasik seperti regresi, SVM, dan pohon keputusan.
    \item PyTorch dikenal dengan grafik komputasi dinamis yang cocok untuk riset dan \textit{prototyping}. 
\end{itemize}

\textbf{Arsitektur TensorFlow} secara umum dapat terdiri dari beberapa komponen utama sebagai berikut. 

\begin{itemize}
    \item \textbf{Tensor}

    Tensor adalah struktur data utama di TensorFlow. Tensor merupakan suatu vektor atau matriks multidimensi yang digunakan untuk menyimpan \textit{input}, parameter, dan \textit{output} model. 

    \item \textbf{Computational Graph}

    TensorFlow membangun \textit{graph komputasi} yang terdiri dari \textbf{\textit{node}} yakni operasi matematis (penjumlahan, perkalian, aktivasi) serta \textbf{\textit{edge}} berupa aliran tensor di antara operasi. 

    \item \textbf{Execution}

    TensorFlow memiliki dua model eksekusi yakni Graph Execution (komputasi dioptimalkan terlebih dahulu) dan Eager Execution (hasil langsung dihitung).
\end{itemize}

\textbf{Ekosistem TensorFlow} mendukung seluruh proses dari pembangunan dan pelatihan model. TensorFlow memiliki beberapa ekosistem di antaranya sebagai berikut. 

\subsection{TensorFlow Core}
API tingkat rendah untuk:
\begin{itemize}
    \item operasi tensor
    \item pembuatan computational graph
    \item kontrol eksekusi
\end{itemize}

\subsection{Keras}
API tingkat tinggi di dalam TensorFlow untuk:
\begin{itemize}
    \item membangun neural network secara sederhana
    \item menyusun layer secara berurutan
    \item training dan evaluasi model
\end{itemize}

\subsection{TensorBoard}
Alat visualisasi untuk:
\begin{itemize}
    \item memantau proses training
    \item melihat grafik loss dan akurasi
    \item melihat struktur model dan graph
\end{itemize}

\subsection{TensorFlow Lite}
Digunakan untuk:
\begin{itemize}
    \item deployment pada perangkat mobile
    \item IoT dan edge device
    \item optimasi model menjadi lebih ringan
\end{itemize}

\subsection{TensorFlow Serving}
Digunakan untuk:
\begin{itemize}
    \item deployment model pada server
    \item layanan API untuk model machine learning
\end{itemize}

\subsection{TensorFlow.js}
Digunakan untuk menjalankan model:
\begin{itemize}
    \item langsung di browser
    \item menggunakan JavaScript
\end{itemize}

\subsection{TensorFlow Hub}
Repositori yang menyediakan:
\begin{itemize}
    \item model siap pakai (pre-trained)
    \item transfer learning
\end{itemize}


\end{document}