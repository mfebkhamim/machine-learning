\documentclass[a4paper,12pt]{article}

\usepackage[margin=2.5cm, top=2cm]{geometry}   % atur margin + jarak atas
\usepackage{setspace}
\usepackage{xcolor}
\usepackage{listings}
\usepackage{titlesec}
\usepackage{array}
\usepackage{tabularx}
\usepackage{indentfirst}
\usepackage{booktabs}
\usepackage[utf8]{inputenc}
\usepackage[T1]{fontenc}
\usepackage{amsmath}

% Pengaturan warna untuk kode Python
\definecolor{codegreen}{rgb}{0,0.6,0}
\definecolor{codegray}{rgb}{0.5,0.5,0.5}
\definecolor{codepurple}{rgb}{0.58,0,0.82}
\definecolor{backcolour}{rgb}{0.95,0.95,0.92}

% agar kolom X bisa vertical centering (middle)
\renewcommand{\tabularxcolumn}[1]{m{#1}}

% kolom X yang center horizontal
\newcolumntype{Y}{>{\centering\arraybackslash}X}

\lstset{
    language=Python,
    backgroundcolor=\color{backcolour},
    commentstyle=\color{codegreen},
    keywordstyle=\color{magenta},
    numberstyle=\tiny\color{codegray},
    stringstyle=\color{codepurple},
    basicstyle=\ttfamily\small,
    breaklines=true,
    showstringspaces=false
}

\titleformat{\title}{\normalfont\Large\bfseries}{}{0pt}{}
\titlespacing*{\section}{0pt}{0.8\baselineskip}{0.6\baselineskip}

\title{\textbf{Machine Learning}}
\author{Mohammad Febryan Khamim}
\date{} % kosongkan tanggal agar tidak muncul

\begin{document}

\maketitle

\section{Pengantar Machine Learning}

\textbf{\textit{Machine Learning}} adalah teknik yang memungkinkan sistem komputer untuk belajar dan mengambil keputusan pada tugas tertentu tanpa harus diprogram secara eksplisit. 

    \begin{quote}
        \textbf{Konsep Sederhana} : \textit{Trial and Error} $\rightarrow$ \textbf{\textit{Machine learning}} memungkinkan komputer untuk meningkatkan performanya untuk menuntaskan tugas tertentu secara otomatis. 
    \end{quote}

\subsection{Karakteristik Utama Machine Learning}

\begin{itemize}
    \item \textbf{Belajar dari data} : Model menemukan pola tersembunyi
    \item \textbf{Perbaikan otomatis} : Kinerja meningkat seiring pertambahan data
    \item \textbf{Generalisasi} : Tidak mengenali pola dalam data tertentu saja
\end{itemize}

\subsection{Kapan Menggunakan Machine Learning?}
    \begin{itemize}
        \item Ketika tugas atau permasalahan mengandung pola tertentu yang terlalu kompleks untuk diselesaikan dengan \textit{rules}
        \item Terdapat data yang tak diketahui hubungan antarfiturnya
        \item Permasalahan mengandung probabilitas dan ketidakpastian
        \item Tugas memerlukan adaptasi atau pengembangan seiring pertambahan waktu. 
    \end{itemize}

\subsection{Jenis-Jenis Machine Learning}
    Terdapat beberapa jenis Machine Learning, di antaranya sebagai berikut. 
    \begin{itemize}
        \item \textbf{Unsupervised Learning} : Data tak berlabel
        \item \textbf{Supervised Learning} : Data berlabel
        \item \textbf{Reinforcement Learning} : Mempelajari data dengan \textit{trial n error} untuk optimalkan \textit{reward}
    \end{itemize}

\subsection{Proses Machine Learning}

$\text{Input data} \rightarrow \text{Algoritma ML} \rightarrow \text{Pelatihan Model} \rightarrow \text{Feedback loop} \rightarrow \text{Proses iterasi} \rightarrow \text{Evaluasi dan generalisasi}$

\section{Supervised Learning}

Dalam \textit{supervised learning}, kita melatih model menggunakan data yang telah memiliki kelas atau label. 
Setelah proses \textit{training}, model diuji dengan \textit{data tes} untuk mengetahui seberapa baik performa prediksi yang dihasilkan.

\subsection{Metrik Evaluasi dan Robustness}
\begin{itemize}
    \item \textbf{Robustness}: Untuk mengukur seberapa \textit{robust} sebuah model, dapat digunakan \textit{Mean Square Error} (MSE).
    \item \textbf{Tipe Tugas}:
    \begin{itemize}
        \item \textbf{Klasifikasi}: Memprediksi label atau kategori data.
        \item \textbf{Regresi}: Memprediksi nilai kontinyu atau jumlah, contohnya prediksi harga rumah berdasarkan fitur-fitur tertentu.
    \end{itemize}
\end{itemize}

\subsection{Pembagian Data}
Terdapat 3 jenis data utama dalam menyelesaikan tugas \textit{Machine Learning}:
\begin{enumerate}
    \item \textbf{Training Data}: Digunakan untuk melatih parameter model.
    \item \textbf{Validation Data}: Digunakan untuk menentukan \textit{hyperparameter} yang perlu disesuaikan (\textit{adjusted}) serta memilih model terbaik.
    \item \textbf{Test Data}: Digunakan untuk mengevaluasi performa akhir dan menentukan metrik evaluasi.
\end{enumerate}

\section{Metrik Evaluasi Klasifikasi}

Evaluasi performa klasifikasi menggunakan beberapa metrik utama: Akurasi, \textit{Recall}, \textit{Precision}, dan \textit{F1-Score}.

\begin{itemize}
    \item \textbf{Akurasi}: Jumlah total jawaban benar dibagi dengan banyaknya jumlah prediksi. Metrik ini baik digunakan pada data yang seimbang (\textit{balanced}), tetapi kurang baik untuk \textit{imbalanced data}.
    \item \textbf{Recall}: Menunjukkan kemampuan model menemukan semua kasus relevan pada data.
    \[ \text{Recall} = \frac{TP}{TP + FN} \]
    \item \textbf{Precision}: Kemampuan model klasifikasi untuk mengidentifikasi titik data yang relevan secara tepat.
    \[ \text{Precision} = \frac{TP}{TP + FP} \]
    \item \textbf{F1-Score}: Metrik campuran optimal (\textit{optimal blend}) antara \textit{precision} dan \textit{recall}.
    \[ \text{F1-Score} = 2 \times \frac{\text{Precision} \times \text{Recall}}{\text{Precision} + \text{Recall}} \]
\end{itemize}

\subsection{Confusion Matrix}
Tabel berikut merepresentasikan \textit{Confusion Matrix} untuk evaluasi prediksi:

\begin{center}
\begin{tabular}{lcc}
\toprule
 & \textbf{Pred. Positif} & \textbf{Pred. Negatif} \\ \midrule
\textbf{Cond. Positif} & TP (True Positive) & FN (False Negative) \\
\textbf{Cond. Negatif} & FP (False Positive) & TN (True Negative) \\ \bottomrule
\end{tabular}
\end{center}

\section{Tugas-Tugas pada Machine Learning}

\subsection{Supervised Learning}
\begin{enumerate}
    \item \textbf{Klasifikasi}: Decision Tree, SVM, Random Forest.
    \item \textbf{Regresi}: Linear Regression, Logistic Regression.
    \item \textbf{Time-series}: LSTM, ARIMA.
\end{enumerate}

\subsection{Unsupervised Learning}
\begin{enumerate}
    \item \textbf{Klasterisasi}: K-means, Hierarchical, Gaussian Mixture.
    \item \textbf{Reduksi Dimensi}: PCA, t-SNE, Autoencoder.
    \item \textbf{Association Rule}: Apriori.
    \item \textbf{Anomaly Detection}: GMM.
\end{enumerate}

\section{Konsep Lanjutan Machine Learning}

\subsection{Bias-Variance Tradeoff}
Ketidakseimbangan antara bias dan varians menentukan kemampuan generalisasi model:
\begin{itemize}
    \item \textbf{Underfitting}: Terjadi saat model terlalu sederhana (\textit{High Bias}).
    \item \textbf{Overfitting}: Terjadi saat model terlalu kompleks dan menghafal \textit{noise} pada data (\textit{High Variance}).
\end{itemize}

\subsection{Teknik Regularisasi}
Digunakan untuk mencegah \textit{overfitting} dengan menambahkan penalti pada fungsi \textit{loss}:
\begin{itemize}
    \item \textbf{Lasso (L1)}: $Loss + \lambda \sum |w|$ (Dapat melakukan seleksi fitur).
    \item \textbf{Ridge (L2)}: $Loss + \lambda \sum w^2$ (Mengecilkan bobot secara merata).
\end{itemize}

\subsection{Cross-Validation}
Teknik untuk mengevaluasi performa model dengan membagi data menjadi $k$ lipatan (\textit{k-fold}). Model dilatih $k$ kali, menggunakan lipatan yang berbeda sebagai data validasi setiap kalinya untuk memastikan hasil evaluasi yang lebih objektif.

\end{document} 