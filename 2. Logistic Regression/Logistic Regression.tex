\documentclass[a4paper,12pt]{article}

\usepackage[utf8]{inputenc}
\usepackage[T1]{fontenc}
\usepackage[margin=2.5cm, top=2cm]{geometry}   % atur margin + jarak atas
\usepackage{setspace}
\usepackage{xcolor}
\usepackage{listings}
\usepackage{titlesec}
\usepackage{array}
\usepackage{tabularx}
\usepackage{indentfirst}
\usepackage{enumitem}
\usepackage{booktabs}
\usepackage{tcolorbox}
\usepackage{xcolor}
\usepackage[utf8]{inputenc}
\usepackage[T1]{fontenc}
\usepackage{amsmath}

% Pengaturan warna untuk kode Python
% --- Konfigurasi Kode Python ---
\definecolor{codegreen}{rgb}{0,0.6,0}
\definecolor{codegray}{rgb}{0.5,0.5,0.5}
\definecolor{codepurple}{rgb}{0.58,0,0.82}
\definecolor{backcolour}{rgb}{0.95,0.95,0.92}

% agar kolom X bisa vertical centering (middle)
\renewcommand{\tabularxcolumn}[1]{m{#1}}

% kolom X yang center horizontal
\newcolumntype{Y}{>{\centering\arraybackslash}X}

\lstset{
    language=Python,
    backgroundcolor=\color{backcolour},
    commentstyle=\color{codegreen},
    keywordstyle=\color{magenta},
    numberstyle=\tiny\color{codegray},
    stringstyle=\color{codepurple},
    basicstyle=\ttfamily\small,
    breaklines=true,
    showstringspaces=false
}

\titleformat{\title}{\normalfont\Large\bfseries}{}{0pt}{}
\titlespacing*{\section}{0pt}{0.8\baselineskip}{0.6\baselineskip}

\title{\textbf{Logistic Regression}}
\author{Mohammad Febryan Khamim}
\date{} 

\begin{document}

\maketitle

\section{Pengantar Logistic Regression}
\textit{\textbf{Logistic Regression}} adalah algoritma \textit{supervised learning} untuk permasalahan klasifikasi. 
Regresi ini menggunakan klasifikasi \textit{binary} sehingga menghasilkan output \textit{binary}. 
\textbf{\textit{Logistic Regression}} menggunakan \textbf{fungsi sigmoid} untuk meng-\textit{convert} input menjadi nilai probabilitas. 

\section{Linear Regression vs Logistic Regression}
Terdapat beberapa perbedaan antara regresi linear dan regresi logistik, di antaranya sebagai berikut. 

\begin{center}
    \begin{tabular}{|c|c|}
    \hline
    \textbf{Regresi Linear} & \textbf{Regresi Logistik} \\ \hline
    Prediksi nilai kontinu & Prediksi kelas kategorikal \\ \hline
    Menggunakan \textit{best-fit line} & Menggunakan kurva sigmoid \\ \hline
    Menyelesaikan masalah regresi & Menyelesaikan masalah klasifikasi \\ \hline
    \end{tabular}
\end{center}

\section{Tipe-Tipe Logistic Regression}

Terdapat beberapa tipe Logistic Regression berdasarkan variabel dependennya, di antaranya: 

\begin{itemize}[label=\textbullet, font=\color{black}]
    \item \textbf{Binomial Logistic Regression (Regresi Logistik Binomial)} \\
    Tipe ini digunakan ketika variabel dependen hanya memiliki dua kategori kemungkinan (biner).
    \begin{itemize}
        \item \textit{Contoh:} Ya atau Tidak, Lulus atau Gagal, 0 atau 1.
    \end{itemize}

    \item \textbf{Multinomial Logistic Regression (Regresi Logistik Multinomial)} \\
    Tipe ini digunakan ketika variabel dependen memiliki tiga atau lebih kategori kemungkinan yang tidak memiliki urutan atau peringkat tertentu.
    \begin{itemize}
        \item \textit{Contoh:} Klasifikasi jenis hewan (Kucing, Anjing, Domba).
    \end{itemize}

    \item \textbf{Ordinal Logistic Regression (Regresi Logistik Ordinal)} \\
    Tipe ini digunakan ketika variabel dependen memiliki tiga atau lebih kategori yang memiliki urutan alami atau tingkatan (peringkat).
    \begin{itemize}
        \item \textit{Contoh:} Penilaian atau rating (Low, Medium, High).
    \end{itemize}
\end{itemize}

\section{Asumsi Regresi Logistik}

\begin{enumerate}
    \item Observasi independen $\rightarrow$ Tak ada korelasi antar sampel
    \item Variabel binary dependen $\rightarrow$ Variabel harus biner, gunakan \textbf{softmax}
    \item Hubungan linearitas antara variabel \& log $\rightarrow$ Transformasi fitur
    \item Tak ada \textit{outlier}
    \item Ukuran sampel besar
\end{enumerate}

\section{Apa Itu Fungsi Sigmoid?}

\begin{itemize}
    \item Sigmoid merupakan fungsi untuk convert output mentah menjadi nilai probabilitas antara $0$--$1$.
    \item Fungsi ini mengonversi sembarang angka menjadi antara $0$--$1$ yang membentuk kurva ``S''.
    \item Dalam \textbf{Logistic Regression}, digunakan \textit{threshold}, umumnya $0{,}5$. 
    \begin{itemize}
        \item Jika sigmoid lebih dari atau sama dengan \textit{threshold} $\rightarrow$ anggap 1
        \item Jika di bawah 0 $\rightarrow$ anggap 0
    \end{itemize}
\end{itemize}

\section{Konsep Matematis Logistic Regression}

Berikut adalah konsep matematis dalam algoritma Logistic Regression:

\begin{enumerate}
    \item Anggap terdapat sebuah fitur input yang direpresentasikan dalam matriks dan variabel dependen $Y$ yang bernilai biner $0$ atau $1$.

    \begin{equation*}
        \begin{bmatrix}
            x_{11} & \cdots & x_{1m} \\
            x_{21} & \cdots & x_{2m} \\
            \vdots & \ddots & \vdots \\
            x_{n1} & \cdots & x_{nm}
        \end{bmatrix}
    \end{equation*}

    dan 

    \begin{equation*}
        Y =
            \begin{cases}
            0, & \text{jika class 1}, \\
            1, & \text{jika class 2}.
            \end{cases}
    \end{equation*} 

    \item Dari input tersebut, maka berlaku kombinasi linear:

    \begin{equation*}
        z = \left( \sum_{i=1}^{n} w_i x_i \right) + b
    \end{equation*}

    dengan:
    \begin{itemize}
        \item $\sum_{i=1}^{n} w_i x_i = \text{bobot / koefisien}$
        \item $z = \text{skor linear}$
        \item $b = \text{intercept}$
        \item $x_i = \text{fitur input}$
    \end{itemize}

    \item Mengubah dari nilai kontinu z menjadi probabilitas antara 0 hingga 1 yang dapat digunakan untuk prediksi kelas

    \textbf{(*) Gunakan fungsi Sigmoid}

    \begin{equation*}
        \sigma(z) = \dfrac{1}{1+e^{-z}}
    \end{equation*}

    Sehingga, diperoleh grafik '\textbf{S-\textit{Shape}}' dengan:

    \begin{itemize}
        \item $\sigma(z) = 1$ saat $z\rightarrow \infty$
        \item $\sigma(z) = 0$ saat $z\rightarrow -\infty$
        \item $\sigma(z)$ selalu di antara 0 hingga 1
    \end{itemize}

    dengan probabilitas kelas dalam \textbf{\textit{Logistic Regression}} dapat dihitung: 

    
    
    \textbf{Persamaan Logistic Regression dan Peluang}

    Berikut adalah penurunan hubungan antara \textit{odds} dan probabilitas:

    \begin{enumerate}
        \item \textbf{Definisi Odds:}
        \begin{equation*}
            \frac{P(x)}{1 - P(x)} = e^z
        \end{equation*}
    
        \item \textbf{Log-Odds (Logit):}
        \begin{align*}
            \log \left( \frac{P(x)}{1 - P(x)} \right) &= \log e^z \\
            \log \left( \frac{P(x)}{1 - P(x)} \right) &= z \\
            &= w \cdot X + b
        \end{align*}
    
        \item \textbf{Penurunan Menuju Fungsi Sigmoid:}
        \begin{align*}
            \frac{P(x)}{1 - P(x)} &= e^{w \cdot X + b} \\
            P(x) &= (1 - P(x)) e^{w \cdot X + b} \\
            P(x) &= e^{w \cdot X + b} - P(x) e^{w \cdot X + b} \\
            P(x) \left( 1 + e^{w \cdot X + b} \right) &= e^{w \cdot X + b}
        \end{align*}
        
        Sehingga diperoleh fungsi probabilitas:
        \begin{equation*}
            P(x) = \frac{e^{w \cdot X + b}}{1 + e^{w \cdot X + b}} \quad \text{atau} \quad P(x) = \frac{1}{1 + e^{-(w \cdot X + b)}}
        \end{equation*}
    \end{enumerate}
    
\end{enumerate}

\newpage

\section{Penerapan pada Program Python}

\subsection{Import Libraries yang Dibutuhkan}
\begin{verbatim}
import pandas as pd
import numpy as np


from sklearn.model_selection import train_test_split
from sklearn.preprocessing import LabelEncoder, StandardScaler
from sklearn.linear_model import LogisticRegression
from sklearn.metrics import accuracy_score, classification_report, confusion_matrix
\end{verbatim}

\subsection{Membaca File CSV dan Informasi pada Data}

\begin{verbatim}
data = pd.read_csv("data.csv")


print(data.head())
print(data.info())
print(data.describe())
\end{verbatim}

\subsection{Prapemrosesan Data} % Cek missing values, mengisi, dan drop features yang ga dibutuhkan

\begin{verbatim}
# Cek missing values
print(data.isnull().sum())


# Mengisi missing values (contoh: numerik dengan mean)
for col in data.select_dtypes(include=np.number).columns:
data[col].fillna(data[col].mean(), inplace=True)


# Drop fitur yang tidak dibutuhkan
data.drop(columns=['ID'], inplace=True, errors='ignore')
\end{verbatim}

\subsection{Konversi Data Kategorikal}

\begin{verbatim}
label_encoders = {}


for col in data.select_dtypes(include='object').columns:
le = LabelEncoder()
data[col] = le.fit_transform(data[col])
label_encoders[col] = le
\end{verbatim}

\subsection{Diperoleh Fitur yang Akan Dipakai}

\begin{verbatim}
X = data.drop(columns=['Target'])
y = data['Target']


print(X.head())
print(y.head())
\end{verbatim}

\subsection{Konstruksi Model}

\begin{verbatim}
# Split data
X_train, X_test, y_train, y_test = train_test_split(
X, y, test_size=0.2, random_state=42
)


# Standarisasi fitur
scaler = StandardScaler()
X_train = scaler.fit_transform(X_train)
X_test = scaler.transform(X_test)


# Inisialisasi model
model = LogisticRegression()
\end{verbatim}

\subsection{\textit{Running} Model}
\begin{verbatim}
model.fit(X_train, y_train)


y_pred = model.predict(X_test)
\end{verbatim}

\subsection{Evaluasi Model}
\begin{verbatim}
accuracy = accuracy_score(y_test, y_pred)
print("Accuracy:", accuracy)


print("\nClassification Report:")
print(classification_report(y_test, y_pred))


print("\nConfusion Matrix:")
print(confusion_matrix(y_test, y_pred))
\end{verbatim}

\newpage

\section{Full Program Logistic Regression}

\begin{lstlisting}[language=Python, caption=Program Logistic Regression Python]

# Import Libraries
import pandas as pd
import numpy as np
import matplotlib.pyplot as plt
import seaborn as sns
%matplorlib inline


# PRA-PEMROSESAN DATA

# Membaca data
data = pd.read_csv(r'E:\Perkuliahan\Karier\13-Logistic-Regression\advertising.csv')

# Menampilkan Informasi Dasar tentang Data
data.head()
data.info()
data.describe()

# EXPLORATORY DATA ANALYSIS (EDA)

# Membuat Histogram untuk Age
sns.set_style('whitegrid')
data['Age'].hist(bins=30)
plt.xlabel('Age')

# Membuat Jointplot antara Age dan Area Income
sns.jointplot(x='Age', y='Area Income', data=data)

# Membuat Jointplot antara Daily Time Spent on Site dan Age
sns.jointplot(x='Age',y='Daily Time Spent on Site',data=ad_data,color='red',kind='kde')

# Membuat jointplot antara Daily Time Spent on Site dan Daily Internet Usage
sns.jointplot(x='Daily Time Spent on Site',y='Daily Internet Usage',data=ad_data,color='green')

# Membuat pairplot dengan hue berdasarkan Clicked on Ad
# Clicked on Ad adalah variabel target
sns.pairplot(data,hue='Clicked on Ad',palette='bwr')

# PELATIHAN MODEL LOGISTIC REGRESSION

# Memisahkan data train dan data test
from sklearn.model_selection import train_test_split
X = data[['Daily Time Spent on Site', 'Age', 'Area Income','Daily Internet Usage', 'Male']]
y = data['Clicked on Ad']

X_train, X_test, y_train, y_test = train_test_split(X, y, test_size=0.33, random_state=42)

# Melatih model Logistic Regression
from sklearn.linear_model import LogisticRegression
logmodel = LogisticRegression()
logmodel.fit(X_train,y_train)

# MELAKUKAN PREDIKSI
predictions = logmodel.predict(X_test)

# Evaluasi Model
from sklearn.metrics import classification_report, confusion_matrix
print(classification_report(y_test,predictions))

\end{lstlisting}

\end{document} 